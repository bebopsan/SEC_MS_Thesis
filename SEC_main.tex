%%%%%%%%%%%%%%%%%%%%%%%%%%%%%%%%%%%%%%%%%


\documentclass[11pt,spanish, oneside]{SETUP/ezthesis} 
%% # Opciones disponibles para el documento #
%%
%% Las opciones con un (*) son las opciones predeterminadas.
%%
%% Modo de compilar:
%%   draft            - borrador con marcas de fecha y sin im'agenes
%%   draftmarks       - borrador con marcas de fecha y con im'agenes
%%   final (*)        - version final de la tesis
%%
%% Tama'no de papel:
%%   letterpaper (*)  - tama'no carta (Am'erica) 
%%   a4paper          - tama'no A4    (Europa)
%%
%% Formato de impresi'on:
%%   oneside          - hojas impresas por un solo lado
%%   twoside (*)      - hijas impresas por ambos lados
%%
%% Tama'no de letra:
%%   10pt, 11pt, o 12pt (*)
%%
%% Espaciado entre renglones:
%%   singlespace      - espacio sencillo
%%   onehalfspace (*) - espacio de 1.5
%%   doublespace      - a doble espacio
%%
%% Formato de las referencias bibliogr'aficas:
%%   numbers          - numeradas, p.e. [1]
%%   authoryear (*)   - por autor y a'no, p.e. (Newton, 1997)
%%
%% Opciones adicionales:
%%   spanish         - tesis escrita en espa'nol
%%
%% Desactivar opciones especiales:
%%   nobibtoc   - no incluir la bibiolgraf'ia en el 'Indice general
%%   nofancyhdr - no incluir "fancyhdr" para producir los encabezados
%%   nocolors   - no incluir "xcolor" para producir ligas con colores
%%   nographicx - no incluir "graphicx" para insertar gr'aficos
%%   nonatbib   - no incluir "natbib" para administrar la bibliograf'ia

%\usepackage[inline]{trackchanges}
\usepackage[finalnew]{trackchanges}
\addeditor{SEC}
\usepackage[compact]{titlesec}
% idioma
\usepackage[utf8]{inputenc}
%\usepackage[USenglish,british,american,australian,english]{babel}

\usepackage{booktabs} %tablas
\usepackage{tabularx}
\usepackage{rotating}  %rotar tablas
%\usepackage[style=mla]{biblatex}  %Manejo de referencias
\usepackage{colortbl} %color tablas
\usepackage{setspace} %espaciado
\usepackage{url}
%\usepackage{hyperref}
% Paquetes de la AMS:
\usepackage{amsmath, amsthm, amsfonts, amssymb}




% %margenes segun n. icontec
% \usepackage{vmargin}
%   \setmarginsrb  { 3.0cm}  % left margin
%                  { 4.0cm}  % top margcm
%                  { 3.0cm}  % right margcm
%                  { 3.0cm}  % bottom margcm
%                  {  10pt}  % head height
%                  {0.25cm}  % head sep
%                  {   9pt}  % foot height
%                  { 0.8cm}  % foot sep

\onehalfspacing
\pagestyle{plain} % numeracion en la parte inferior

% Default fixed font does not support bold face
\DeclareFixedFont{\ttb}{T1}{txtt}{bx}{n}{8} % for bold
\DeclareFixedFont{\ttm}{T1}{txtt}{m}{n}{8}  % for normal

\setlength{\parindent}{0pt}
\setlength{\parskip}{2.0ex plus0.5ex minus0.2ex}

\author{Santiago Echeverri Chacón}
\authoremail{sechev14@eafit.edu.co}
\supervisor{René Restrepo Gómez}
\supervisoremail{rrestre6@eafit.edu.co}
\degree{MÁGISTER EN FÍSICA APLICADA}
\program{Maestría en Física Aplicada}
\institution{Universidad EAFIT}
\faculty{Escuela de Ciencias}
\department{Departmento de Ciencias Físicas}
\title{GENERACIÓN Y CARACTERIZACIÓN DE VÓRTICES
                  ÓPTICOS MEDIANTE MODULADORES ESPACIALES DE LUZ}


\begin{document}

\include{Preface/titlepage}

\frontmatter

\clearpage

% \begin{savequote}[100mm]
% \emph{``The first principle is that you must not fool yourself - and you are the easiest person to fool...''} \\
% \qauthor{Richard P. Feynmann}
% \end{savequote}

\vspace*{5cm}
\begin{flushright}
\begin{tabular}{p{10cm}r}
\emph{``The first principle is that you must not fool yourself - and you are the easiest person to fool...''} \\
\textbf{Richard P. Feynman}
\end{tabular}
\end{flushright}
\thispagestyle{empty}

\include{Preface/dedication}
% Thesis Acknowledgements ------------------------------------------------


% Opening of the acknowledgements

%Sort version
%this creates the heading for the acknowlegments
\begin{acknowledgements} 
%Long version
%uncommenting this line, gives a different acknowledgements heading
%\begin{acknowledgementslong} 

Muchas gracias. A mis padres por haberme dado la oportunidad y soporte
para llegar hasta aquí. A Satya por su cariño y compañía. A René y Luciano, por su
dirección, amistad, apoyo, y por aceptarme en su grupo. A Nestor, por
sus excelentes ideas y por su muy acertada y equilibrada asesoría
académica y personal. A Carlos y 
Camilo por su ayuda y trabajo duro en el proyecto.\\
\\
\\
\\
Asimismo, agradezco a la Universidad EAFIT y la Dirección de
investigación que me otorgó una beca en el marco de la convocatoria
para proyectos internos y a Colciencias y su programa Jóvenes
Investigadores. Sin estas dos instituciones como soporte habría sido
muy difícil cursar mis estudios de maestría y a la vez generar los
resultados de investigación que se presentan en esta disertación. 

\begin{flushright}

Santiago

% Moth and year
\monthname \ \the\year



% Signature figure

%\begin{figure}[htbp!]
%\end{figure}
%\includegraphics{signature}%



\end{flushright}



%Closing of the acknowledgements
%Sort version
\end{acknowledgements}
% Long version
%\end{acknowledgementslong}

% ------------------------------------------------------------------------




% Thesis Abstract -----------------------------------------------------


%\begin{abstractslong}    %uncommenting this line, gives a different abstract heading


\begin{resumen}        %this creates the heading for the abstract page
% Pon tu resumen aquí.

Se desarrolló un método novedoso de reconstrucción de fase basado en
la técnica de Diversidad de Fase (PD) para la caracterización de
aberraciones ópticas en sistemas 4F con iluminación coherente. La
novedad del metodo resulta del uso de máscaras de fase espiral como
diverisdad de fase, que producen haces portadores de momento orbital
distinto de cero. Nuestras simulaciones y experimentos demuestran que
el uso de haces con dislocaciones mejora la precición de la
aproximación y permite generar haces de alta calidad. 


\end{resumen}



%\end{abstractlongs}


% ---------------------------------------------------------------------- 


%: ----------------------- contents ------------------------

\setcounter{secnumdepth}{5} % organisational level that receives a numbers
\setcounter{tocdepth}{5}    % print table of contents for level 3
\tableofcontents
%: ----------------------- list of figures/tables ------------------------
\listoffigures

% \include{./src/Aknowledgments}
% \include{./src/Introduction}
% \include{./src/Problem}

% \include{./src/Electromagnetic_waves_in_periodic_media}
% \include{./src/Finite_element_method}
% \include{./src/Implementation}
% \include{./src/Results}
% \include{./src/Conclusions}



%: ----------------------- glossary ------------------------

% this file is called up by thesis.tex
% content in this file will be fed into the main document

% Glossary entries are defined with the command \nomenclature{1}{2}
% 1 = Entry name, e.g. abbreviation; 2 = Explanation
% You can place all explanations in this separate file or declare them in the middle of the text. Either way they will be collected in the glossary.
% required to print nomenclature name to page header
%\markboth{\MakeUppercase{\nomname}}{\MakeUppercase{\nomname}}

% ----------------------- contents from here ------------------------
%
\newacronym{OAM}{OAM}{Momento Angular Orbital - Optical Angular
Momentum}
\newacronym{VOs}{VOs}{Vórtices Ópticos - Optical Vortices (OV)}
\newacronym{LG}{LG}{Laguerre-Gauss}
\newacronym{SLM}{SLM}{Modulador Espacial de Luz - Spatial Light
Modulator}
\newacronym{LCD}{LCD}{Pantalla de Cristál Líquido - Liquid Crystal Display}
%\nomenclature{OAM}{Momento Angular Orbital - Optical Angular Momentum}
% \nomenclature{VO}{Vórtices Ópticos - Optical Vortices (OV)}
\newacronym{LCs}{LCs}{Cristales Líquidos - Liquid Crystals}
\newacronym{TN-LCD}{TN-LCD}{Pantalla de Cristál Líquido tipo Nemático Retorcido -Twisted Nematic Liquid Crystal Display}
\newacronym{RMS}{RMS}{Error cuadrático medio - Root Mean Square}
\newacronym{AO}{AO}{Óptica Adaptativa - Adaptive Optics}
\newacronym{NI-WFS}{NI-WFS}{Sensado de Fase No Interferométrico - Non Interferometric Wavefront Sensing} 
\newacronym{PSF}{PSF}{Función de Dispersión de Punto - Point Spread
Function}
\newacronym{PD}{PD}{Diversidad de Fase - Phase Diversity}
\newacronym{GSAs}{GSAs}{Métodos de Búsqueda del Gradiente - Gradient
Search Algorithms}
\newacronym{GS}{GS}{Método de Gerchgerg-Saxton}
\newacronym{APSF}{APSF}{Función de Dispersión de Punto de Amplitud - Amplitude Point Spread Function} \newacronym{FT}{FT}{Transformada de Fourier - Fourier Transform}
\newacronym{OTF}{OTF}{Función de transferencia óptica - Optical transfer function}
\newacronym{GP}{GP}{Pupila Generalizada - Generalized Pupil}
\newacronym{GPU}{GPU}{Unidad de Procesamiento Gráfico -Graphics
Processing Unit}
\newacronym{PSG}{PSG}{Generador de Estados de Polarización -
Polarization State Generator}
\newacronym{PSD}{PSD}{Detector de Estados de Polarización -
Polarization State Detector}
% \nomenclature{AIV}{Armado, Integración y Verificación}
% \nomenclature{ESPI}{Interferometría de Patrones de Speckle Elecrónica}
% \nomenclature{SLM}{Modulador Espacial de Luz}
% \nomenclature{SH}{Shack-Hartmann}
% \nomenclature{LC}{Cristal Líquido}
\newacronym{CCD}{CCD}{Dispositivo de Carga acoplada}
% \nomenclature{RMSE}{Error cuadratico medio}
% \nomenclature{SNR}{Coeficiente señal-ruido}
% \nomenclature{CTE}{Coeficiente de Expansión Térmica}
% \nomenclature{HDR}{Alto Rango Dinámico}
% \nomenclature{TVC}{Cámara de Termo-Vacío}
% \nomenclature{FEA}{Análisis de Elementos Finitos}
% \nomenclature{FEM}{Modelo de Elementos Finitos}
% \nomenclature{TUM}{Technische Universit\"{a}t M\"{u}nchen - Universidad Tecnológica de Munich }
% \nomenclature{OAM}{Momento Orbital Angular}
% \nomenclature{VCSEL}{Láser Emisor de Superficie de Cavidad Vertical}
% \nomenclature{WCOG}{Método de Compensación del Centro de Gravedad}
% \nomenclature{IWCOG}{Método Iterativo de Compensación del Centro de Gravedad}
% \nomenclature{FWHM}{Anchura Total a la mitad del Máximo}
% \nomenclature{SPT}{Transformada Espiral de Fase}
% \nomenclature{LK}{Método Lucas-Kanade}
% \nomenclature{LINES}{Laboratorio de Instrumentación Espacial}
% \nomenclature{INTA}{Instituto Nacional de Técnica Aeroespacial}
% \nomenclature{LCVR}{Retardadores Variables de Cristal Líquido}
% \nomenclature{FDT}{Full Disk Telescope - Telescopio de Disco Completo}
% \nomenclature{HRT}{High Resolution Telescope - Telescopio de Alta Resolución}
% \nomenclature{RFM}{Refocus Mechanism}
% \nomenclature{PHI}{Polarimetric and Helioseismic Imager}
% \nomenclature{BS}{Divisor de haz - Beam Splitter}
% \nomenclature{IAC}{Instituto de Astrofísica de Canarias}
% \nomenclature{GTC}{Gran Telescopio de Canarias}
% \nomenclature{ESA}{Agencia Espacial Europea}
% \nomenclature{PMP}{Polarization Modulation Package - Paquete de Modulación de la Polarización}
% \nomenclature{CMOS}{Semiconductor Complementario de Óxido Metálico}
% \nomenclature{APAN}{Nemático Antiparalelo - Anti-Parallel Nematic}
% \nomenclature{LBD}{Globos de Larga Duración - Long Duration Balloon}
% \nomenclature{NASA}{Administración Nacional de Aeronáutica y Espacio de USA - National Aeronautics and Space Administration}
% \nomenclature{GACE}{Grupo de Astronomía y Ciencias del Espacio}
% \nomenclature{IAA}{Instituto de Astrofísica de Andalucía}
% \nomenclature{FOV}{Campo de Visión - Field of View}
% \nomenclature{PSF}{Función de Punto Esparcido - Point Spread Function}
% \nomenclature{WFE}{Error del frente de onda - Wavefront error}
% \nomenclature{MTF}{Modulación de la función de transferencia óptica - Modulation Transfer Function}
% \nomenclature{PD}{Diversidad de Fase - Phase Diversity}
% \nomenclature{WHT}{William Herschel Telescope}
% \nomenclature{OGS}{Optical Ground Station}
% \nomenclature{FPGA}{Field-programmable gate array}
% \nomenclature{ENO}{Observatorios Europeos del Hemisferio Norte - European Northern Observatory}
% \nomenclature{AO}{Óptica Adaptativa - Adaptive Optics}
% \nomenclature{GS}{Método de Gerchgerg-Saxton}
% \nomenclature{SBMIR}{Single-beam Multiple-Intensity Reconstruction Technique}
%\nomenclature{ASD}{Descomposición espectral angular - Angular Spectrum Descomposition}

\printnomenclature % [] = distance between entry and description


\label{sec:glossary} % target name for links to glossary

%: --------------------------------------------------------------
%:                  MAIN DOCUMENT SECTION
% --------------------------------------------------------------

% the main text starts here with the introduction, 1st chapter,...
\mainmatter
\pagestyle{fancy}

%: ----------------------- introduction ------------------------
% introduction

% this file is called up by thesis.tex
% content in this file will be fed into the main document

%------------------------------------------------------------------------- 

\chapter{Introducción}
\label{cha:Introduccion}

\graphicspath{{Figures/intro_img/}{../Figures/intro_img/}}

Como es bien sabido, la luz transporta energía; esto se hace evidente al
comparar las temperaturas en el día y en la noche o al iluminar una celda
fotovoltáica. En su representación cuántica, la luz está
compuesta por partículas sin masa llamadas fotones. Al no tener masa,
su energía está directamente asociada a su momento, y el momento de 
los fotones así como el de otras partículas en la mecánica cuántica puede ser tanto
lineal como angular. El momento angular se compone a su vez de dos
contribuciones, la de spin y la orbital. Desde un punto de vista 
macroscópico, el momento angular de spin se asocia con la polarización
de la luz, es decir con la dirección de oscilación de los campos
eléctrico y magnético con respecto a un eje coordenado. Asimismo, el
momento angular orbital (OAM) se asocia con las distribuciones
espaciales de la amplitud y la fase, tal y como se observan
en un plano perpendicular a la propagación de la luz. Para aclarar esta idea
comparemos dos haces polarizados linealmente, uno con OAM
cero, y el otro con OAM +1. El haz de luz que carece de momento
angular orbital presenta una distribución de fase constante. Si éste tiene una distribución de amplitud
Gaussiana, al ser enfocado por una lente, en un plano de
observación veremos que la distribución de intensidad está dada por una función de
Airy como la que se ilustra en la figura \ref{fig:oam_intro}c). 
%\ref{fig:oam0intro}  

lp;./['%\begin{figure}[h!]
%\centering
%\includegraphics[scale=.33]{img/oam0Intro}
%\caption{a) Amplitud normalizada de un haz plano circular con OAM 0. b)
%Mapa de fase envuelta de $-\pi$ a $\pi$ radianes del mismo haz. c)
%Intensidad normalizada y corte transversal de un haz Gaussiano luego de
%ser enfocado.}
%\label{fig:oam0intro}
%\end{figure}

 Por el contrario, el haz con OAM +1 posee una distribución
 de fase helicoidal donde el valor de la fase varía azimutalmente
 desde $\pi$ a $-\pi$ radianes como se muestra  
en la  figura  \ref{fig:oam_intro}b). Haces con distribuciones de fase de este tipo poseen una
indeterminación de la fase en el centro dado que en la
coordenada $r=0$ confluyen fotones con todos los valores posibles de fase. La
consecuencia directa de la indeterminación en este tipo de puntos es
la ausencia de luz por efecto de superposición. Si, como en el caso
anterior, observamos la intensidad en un plano de enfoque veremos
perfil con forma de dona como la de la figura
\ref{fig:oam_intro} d). 


\begin{figure}[h!]
\centering
\includegraphics[scale=.33]{oam_Intro}
\caption{ Las figuras a) y b) representan mapas de fase de haces con
  OAM 0 y +1 definidos en el intervalo $[- 
  \pi,\pi]$ . Las intensidades correspondientes luego de enfocar los haces en
  un plano de observación se muestran en las figuras c) y d).}
\label{fig:oam_intro}
\end{figure}

Por su naturaleza rotacional, los puntos alrededor de los cuales la fase
varía de $-\pi$  a $\pi$ se conocen como \textbf{vórtices ópticos} (VO), y
están presentes siempre que haya haces con momento angular 
orbital distinto de cero. Por otra parte, de forma similar a cómo se
describe la amplitud en haces con OAM cero como ``Gaussiana'', 
los haces con momento angular distinto de cero se describen
matemáticamente como haces \textbf{``Laguerre-Gauss'' (LG)}. Esto se debe a que
soluciones de la ecuación de onda en coordenadas
cilíndricas incluyen no sólo una componente de amplitud Gaussiana, sino
también una dependencia radial y azimutal descrita por polinomios de
Laguerre, con los cuales se pueden representar vórtices ópticos de fase
y amplitudes del tipo dona.       
El estudio, y el desarrollo de aplicaciones sobre los haces Laguerre-Gauss y por consecuencia, de los VO , requiere entonces de la
capacidad de manipular el OAM de haces de luz.  \\

El momento angular orbital añade un grado de libertad al
conjunto de propiedades que pueden ser manipuladas y que caracterizan
a la luz, en particular: la polarización o espin, la coherencia, el
espectro y la cantidad de energía. Siendo así, la posibilidad de manipular el
momento angular orbital abre camino a un amplio rango de aplicaciones
en numerosas áreas de la ciencia y la tecnología, tanto en el mundo
microscópico (células y micromanipulación) como en el macroscópico
(astronomía y telecomunicaciones).  

Por listar brevemente algunas aplicaciones de los haces
con OAM distinto de cero se pueden mencionar: El uso de OAM en
telecomunicaciones ópticas como una nueva variable para 
multiplexación de señales en fibra y en espacio libre 
\citepInt{Lin2007,Gibson2004,Fontaine2012,Gibson2004,Bozinovic2013}. En microscopía
óptica para resaltar bordes de muestras biológicas transparentes \citepInt{Jesacher2005, Bouchal2012}, e identificar
curvaturas de objetos de fase por medio de interferometría espiral
\citepInt{Furhapter2005}. Además, es una herramienta esencial para la
manipulación de objetos en la escala micro al ser usados como pinzas ópticas capaces
de atrapar y mover partículas \citepInt{Grier2003}. Se espera también
un avance importante en el campo de la computación cuántica vía
entrelazamiento cuántico de OAM en fotones \citepInt{Mair2001}. Fuera de
las anteriores, cabe destacar algunas de las patentes relacionadas
con el tema como: aplicaciones en imagenología médica de resonancia magnética
n uclear \citepInt{Elgort}, y teledetección de objetivos militares
\citepInt{Schmitt}. También han sido patentadas herramientas y métodos
para micromanipulación de partículas microscópicas\citepInt{Grier}, con
posibles aplicaciones en bombas peristálticas para microfluidos
\citepInt{Guzzinati2014}. Para concluir, cabe mencionar  que hoy
en día la radiación óptica no es la única que está siendo usada  
para la propagación del momento angular orbital; destacan trabajos en
los cuales se utilizan los regímenes de ondas de radio
\citepInt{Thide2007}, rayos X \citepInt{Sasaki2008}, e inclusive haces de
electrones \citepInt{Guzzinati2014} para transmitir OAM. \\

Las referencias y ejemplos mencionados respaldan e ilustran el
intenso interés que se ha generado sobre el tema en la 
comunidad científica, y en particular en las áreas de óptica
aplicada y fotónica.  En Colombia, el tema de los vórtices ópticos es
un area  incipiente pero fértil. A nivel nacional se destaca una primera
iniciativa teórica por parte del grupo de óptica e información
cuántica de la  Universidad Nacional sede
Bogotá, en la cual se estudió la propagación de haces con OAM distinto de
cero en elementos ópticos conocidos como axicones
\citepInt{Guzman2009}. Asimismo, en el grupo de óptica y tratamiento de
señales de la Universidad Industrial de Santander han trabajado en el diseño de un codificador optoelectrónico
basado en el momento angular\citepInt{CristianAcevedo2012,Meza2013}. Es,
sin embargo en el ámbito regional de Antioquia en el cual se
concentra la mayor cantidad de esfuerzos en Colombia.  El grupo de
Óptica y Procesamiento Opto-digital de la Universidad 
Nacional sede Medellín desarrolló un sistema de pinzas ópticas para la
manipulación de microsistemas \citepInt{Alvarez2011}, mientras que el grupo de Óptica y Fotónica
de la Universidad de Antioquia ha estudiado la Multiplexación de
Información Encriptada y Codificación con Momento Angular
Orbital \citepInt{CarlosAndresRios2010}, así como la generación experimental de
vórtices ópticos con moduladores de transmisión
\citepInt{DavidMuneton2012,Rueda2013}. Además de los esfuerzos
de cada institución, destaca el trabajo colaborativo que se ha afianzado en el marco de convenios de
cooperación tales como el proyecto interinstitucional titulado:\\
\textit{Aberraciones ópticas en haces Laguerre-Gaussianos: corrección
  y aplicaciones metrológicas}. \\Este es un proyecto cuya duración es  de 24 meses, que
comenzó a ejecutarse el 5 de agosto de 2013 y que culminará el 5 de
agosto de 2015. Se desarrolla con la participación de grupos de la
Universidad EAFIT, la Universidad de Antioquia, el Centro de
Investigaciones Ópticas de Argentina, el Politécnico Colombiano Jaime
Isaza Cadavid, y el Instituto Tecnológico Metropolitano. 

De proyectos como este, se ha formado una red de grupos interesados
específicamente en el estudio de VO. En particular,
la cooperación entre algunos de ostos grupos derivó en trabajos en los
cuales se estudió el efecto de la birrefringencia inducida por
cristales birrefringentes en vórtices ópticos \citepInt{Gomez2012a}, y la
posibilidad de generar vórtices con una cantidad reducida de niveles
de gris en moduladores de transmisión \citepInt{Rueda2013}. De forma
similar, la Universidad EAFIT, a través de su grupo de Óptica Aplicada
y en cooperación con el Centro de Investigaciones Ópticas de
Argentina,  ha contribuido con el desarrollo de técnicas metrológicas
computacionales basadas en el estudio de  vórtices en patrones de
speckle
\citepInt{Angel-Toro2012,Angel-Toro2012a,Angel-Toro2013,Sierra-Sosa2013,Sierra-Sosa2013b}. 


Con la iniciativa de adquirir las capacidades técnicas y experimentales
necesarias para el desarrollo de aplicaciones metrológicas de vórtices ópticos, el grupo de Óptica
Aplicada de la Universidad EAFIT ha abierto dos proyectos internos, y
ha sido merecedor de una beca del programa Jóvenes Investigadores de Colciencias,
convocatoria 645 a cursar en el 2015.  
Las prioridades del grupo, y asimismo los temas de trabajo de estos
dos proyectos son: 
\begin{itemize}
\item El desarrollo de aplicaciones metrológicas de haces Laguerre
  Gauss. 
\item La implementación de técnicas basadas en los haces con OAM
  distinto de cero para instrumentos de microscopia de objetos de fase. 
\end{itemize}

Es, en este contexto, que desde Julio del 2013 he venido realizando mi labor de
investigación en la línea de metrologia óptica del grupo de Óptica Aplicada de la Universidad EAFIT
en el proyecto interinstitucional  \textit{Aberraciones ópticas en haces Laguerre-Gaussianos: corrección
  y aplicaciones metrológicas} gozando del beneficio de un beca de Maestría. 

% % duración del programa. Así como realizar mi trabajo de grado en armonía con
% % el mismo. 

% %Este trabajo se plantea como un desarrollo complementario de la línea
% %de investigación del grupo en la cual he venido trabajando

Mediante la presente propuesta de Trabajo de Grado se busca proponer
un proyecto final de maestría con el cual se pueda concluir y documentar
un proceso académico e investigativo de dos años que va a permitirle al
grupo de Óptica Aplicada abrir nuevas areas de trabajo. 
Específicamente, se propone terminar de desarrollar las capacidades técnicas y
habilidades necesarias para la manipulación del OAM y la generación de
VO con miras a la exploración de nuevas aplicaciones de VO en la
metrología óptica y la microscopía. \\


En adelante, en la sección \ref{sec:planteamiento} se presentará el
planteamiento del problema. En la sección \ref{sec:marco_teorico}, se presentará  un breve marco
teórico y estado del arte que dará soporte a las proposiciones del planteamiento del problema.
% Adicionalmente se abordará una breve sección
% de análisis de riesgos (\ref{sec:analisis}) en la cual se pretende evaluar las posibles
% limitaciones y fuentes de retraso o aplazamiento. 
Y en la sección \ref{sec:objetivos} se presentará el objetivo general y enumerarán
los objetivos específicos que deben ser cumplidos al finalizar el
proyecto. Luego, en la sección \ref{sec:metodologia} se hará una descripción de las
actividades a realizar para la consecución de los objetivos
específicos, y se presentará un cronograma (sección \ref{sec:cronograma})  que
establece los tiempos en los cuales se deben cumplir. Finalmente, se
presenta una descripción de los recursos necesarios
para llevar a cabo el proyecto y se listan las referencias
bibliográficas consultadas. 

\section{Motivación y objetivos\label{sec:motiv}}


%------------------ESTADO_DEL_ARTE---------------
\section{Estado del Arte\label{sec:estadoArte}}




%-----------------ESTRUCTURA-------------------
\section{Estructura\label{sec:estructura}}

El texto principal de este trabajo, está dividido en 2 partes temáticas que agrupan los Capítulos. A continuación, se presenta la estructura general de la disertación por Partes y Capítulos: \\


\textbf{Parte \ref{ParteI}: Implementación de una plataforma para la caracterización de un SLM}



\textbf{Parte \ref{ParteII}: Caracterización y corrección de aberraciones de VO}


\textbf{Parte \ref{ParteIV}: Recuperación de fase con métodos no interferométricos}


\newpage
\pagebreak[4]
\bibliographystyleInt{unsrtnat}
\bibliographyInt{References/Int}






	
%\include{1_introduction/1_motivation}	

% Interferometria y HDR
\part{Implementación de una plataforma para la caracterización de un SLM\label{ParteI}}


%\cleardoublepage \phantomsection
%\addcontentsline{toc}{chapter}{References}

\end{document}