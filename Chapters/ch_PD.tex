% this file is called up by thesis.tex
% content in this file will be fed into the main document

%------------------------------------------------------------------------- 

\chapter{Caracterización de aberraciones en Vórtices Ópticos}
\label{cha:Car_intro}
\graphicspath{{Figures/chPD_img/}{../Figures/chPD_img/}}
\lhead{Caracterización de aberraciones en Vórtices Ópticos:
  \textit{Introducción}} % This is for the header on each page -
                         % perhaps a shortened title
\section{Introducción}
En capítulos anteriores ha quedado claro que para producir VO es
necesario contar con un sistema óptico en el cual sea posible
manipular con precisión la fase de un frente de onda.  Asimismo, se
presentó un montaje experimental en el cual logramos generar VO a
partir del uso de dispositivos difractivos conocidos como SLMs. 
No obstante, los VO obtenidos distan de ser de suficiente calidad como
para ser usados en aplicaciones científicas o tecnológicas. 

Esta segunda parte de la tesis abarca el trabajo que se realizó para
mejorar la calidad óptica de nuestro montaje con el fín de mejorar los
VO que se obtuvieron en la parte anterior. 

\section{Estado del Arte}
\label{sec:ChPD_estado_del_arte}
\lhead{Caracterización de aberraciones en Vórtices Ópticos: \textit{Estado
    del Arte}}

Los sistemas ópticos formadores de imágen que se encuentran en
aplicaciones de la vida real están sujetos a aberraciones de fase que
limitan su resolución. Es por ello que en la industria y en laboratorios se hace un gran esfuerzo para
detectar aberraciones y corregirlas vía Óptica Adaptativa (AO) \citepChPD{Kubby2013} o por
medio de técnicas digitales posteriores a la adquisición
\citepChPD{Korkiakoski2012}. 

Las aberraciones ópticas en un sistema formador de imagen pueden
proceder de fuentes intrínsecas tales como imperfecciones en el
diseño, los materiales, la manufactura o la alineación de los
elementos que los componen. O de fuentes extrínsecas como variaciones
en el índice de refracción de muestras microscópicas y turbulencia atmosférica en
imágenes capturadas usando telescopios. La presencia de aberraciones
del último tipo en imágenes procedentes de telescopios terrestres, y
la dificultad de modificar los sistemas para incluir brazos de referencia han sido la motivación para
el desarrollo de varias técnicas de Sensado de Fase no
Interferométricas (NI-WFS). La técnica de Diversidad de Fases o Phase
Diversity (PD) pertenece a una familia de NI-WFS conocida como de
Reconstrucción de Fase o Phase Retrieval. A diferencia de técnicas
directas que requieren de óptica y sensores adicionales como los
sistemas que usan sensores Shack-Hartman, las técnicas de Phase
Retrieval consisten en la determinación de la fase de una función
compleja a partir de medidas de su magnitud usando 
información a priori de la función o de su transformada \citepChPD{Fienup1993}. 
Específicamente, la técnica de reconstrucción  PD ha sido usada
exitosamente en el contexto de sistemas de AO para incrementar la resolución de sistemas ópticos
tales como el Telescopio Espacial Hubble \citepChPD{Fienup1993}, y en
post procesamiento de imágenes de astronomía en las cuales la
resolución es crítica. Dos casos muy relevantes son el estudio de
manchas solares y la detección de planetas extrasolares \citepChPD{Lofdahl1994,Bonet2005,Korkiakoski2012,Sauvage2007,Sauvage2012}. 

Así como con otras técnicas desarrolladas para aplicaciones en
astronomía, los métodos de reconstrucción de fase como el
Gerchberg-Sachston (GS) y PD han migrado a aplicaciones en el
laboratorio, y más específicamente a aplicaciones en microscopía de
fase \citepChPD{Jesacher2007,Camacho2010,Kner2013a}. Tal es el caso
del trabajo de \citetChPD{Jesacher2007} que implementó una versión del
método GS para la optimización de pinzas ópticas utilizadas como
iluminación en sistemas de microscopía de contraste de fase espiral. 

En este capítulo se presenta un método novedoso de reconstrucción de
fase del tipo PD inspirado en la aplicación antes mencionada, y por
medio del cual fue posible detectar y corregir las aberraciones ópticas
del sistema generador de VO presentado en el capítulo
\ref{cha:Gen_intro}.  A continuación se presenta el marco teórico que
soporta la implementación del método. En la sección
\ref{sec:ChPD_materiales_y_metodos} se presenta el 

\section{Marco Teórico}
\label{sec:ChPD_marco_teorico}

\section{Materiales y Métodos}
\label{sec:ChPD_materiales_y_metodos}

\newpage
\pagebreak[4]
\bibliographystyleChPD{unsrtnat}
\bibliographyChPD{References/Ch_PD}