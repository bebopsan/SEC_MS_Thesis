
% this file is called up by thesis.tex
% content in this file will be fed into the main document

%: ----------------------- introduction file header -----------------------

\chapter{Conclusiones y perspectivas}
\label{cha:Conclusiones}

% the code below specifies where the figures are stored
%\ifpdf
%    \graphicspath{{5_conclusion/figures/PNG/}{5_conclusion/figures/PDF/}{5_conclusion/figures/}}
%\else
 %   \graphicspath{{5_conclusion/figures/EPS/}{5_conclusion/figures/}}
%\fi
En el marco de esta tésis se implementó un sistema para la generación
de vórtices ópticos por medio de SLMs de trasmisión y un método para
la caracterización y corrección de aberraciones ópticas en haces Laguerre-Gausianos. 
La combinación de ambos cumple con los objetivos del proyecto de grado
y de los proyectos internos del grupo de investigación en Óptica
Aplicada de la Universidad EAFIT. Asimismo, le deja al laboratório de
Fotónica un conjunto de herramientas para el
desarrollo de técnicas de micoscopía basadas en iluminación con haces
portadores de OAM. 
Específicamente, 
\begin{itemize}
\item Se mostró el resultado de una labor investigativa con la cual
  fue posible establecer un marco conceptual y teórico para la
  caracterización y puesta a punto de un modulador espacial de luz de
  trasmisión basado en cristales líquidos del tipo twisted nematic. 
\item Se presentó un sistema automatizado para la caracterización de
  pantallas de cristál líquido que se compone de una parte física, que
  involucra cuatro rotadores ópticos mecatrónicos, y una parte de
  software que adquiere los datos y los procesa para obtener las
  matrices de Jones que describen el elemento birrefringente para cada
  nivel de gris. 
\item Asimismo, se desarrolló y puso en proceso de registro una aplicacion de software en la
  plataforma Matlab$\circledR$ para la
  generación de máscaras de fase arbitrarias a ser proyectadas en el
  SLM. Esta aplicación permite:
  \begin{itemize}
    \item Crear máscaras de fase espiral de carga entera arbitraria
      sumadas a:
      \begin{itemize}
        \item Lentes.
        \item Rejillas de difracción de varios tipos.
        \item Aberraciones ópticas compuestas a partir de polinomios
          de Zernike.
      \end{itemize}
    \item Discretizar las máscaras de fase en la cantidad de niveles
      deseados, y asignando valores predeterminados a cada uno. 
  \end{itemize}
% \item Por otra parte, se propuso un método novedoso para la
%   caracterización de SLMs basado en el análisis de desplazamiento de
%   franjas en un interferómetro con brazos que no comparten el mismo
%   estado de polarización. Este método ha sido demostrado en
%   simulaciones y nos encontramos en el proceso de corroboración
%   experimental para validarlo. 
 \item Utilizando una configuración de estados de polarización que
   producen alta modulación de fase y baja modulación de amplitud se
   generaron VOs en un sistema óptico 4F usando dos tipos distintos de
   máscaras de fase. Se concluyó que para obtener VOs de calidad es
   prefereible usar máscaras del tipo tenedor sobre máscaras espiral y
   se detectó que aún con buena modulación no se corrigen del todo las aberraciones. 
\item Dado lo anterior se desarrolló un método traído de aplicaciones
  en astronomía para la detección y corrección de aberraciones ópticas en haces con vorticidad óptica.
\item Este método fue validado mediante numerosas simulaciones, y
  experimentos y se propuso como la base para un instrumento que puede
  ser usado en aplicaciones metrológicas. 
\end{itemize}

Finalmente, concluimos que hemos desarrollado la capacidad para
  detectar y corregir aberraciones ópticas en sistemas formadores de imagen
  iluminados con haces portadores de OAM. Adicionalmente, esto abrió una
  línea de trabajo conjunto con el Laboratorio de Instrumentación
  Espacial (LINES) del Instituto Nacional de Técnica Aeroespacial de
  España (INTA), en la cual, se espera implementar el método desarrollado
  en una posible plataforma para la caracterización de aberraciones en
  sistemas formadores de imágen complejos, típicos de la industria
  aeroespacial. Nuestra hipótesis es que el método de PD mejorado con
  VOs puede desempeñarse mejor que otras alternativas no interferométricas para
  el sensado de aberraciones.

\section{Productos desarrollados en el transcurso del proyecto}

En el transcurso del proyecto titulado \textit{Aberraciones ópticas en haces Laguerre-Gaussianos: corrección
  y aplicaciones metrológicas} se desarrollaron los siguientes
productos:

\begin{itemize}
\item El lanzamiento de un artículo científico del tipo ``Letters'' a una revista de alto
  impacto en el cual se describe un método novedoso para la detección
  y corrección de aberraciones. Este artículo se titula: \textbf{Optical Vortex-Enhanced coherent illumination phase diversity for phase retrieval on general imaging systems}.
  
\item Un programa con interfáz gráfica, desarrollado en Python para la
  visualización de estados de polarización arbitrarios y la traducción entre distintas
  notaciones. Este programa fue presentado en el evento internacional
  FOCUS Latinoamérica como una ponencia de Poster y se encuentra en proceso de registro de
  software ante la Dirección Nacional de Derecho de Autor. 
\item Una plataforma computacional desarrollada en el entorno
  Matlab$\circledR$ para la generación de Máscaras de fase arbitrarias
  a ser proyectadas en un SLM que se encuentra en proceso de registro
  de software ante la Dirección Nacional de Derecho de Autor. 
\item Un sistema automatizado de toma de medidas para la
  caracterización de elementos ópticos birrefringentes que consiste
  en:
  \begin{itemize}
  \item  Cuatro monturas motorizadas de elementos polarizadores
    controladas desde un microcontrolador Arduino.
   \item Una plataforma de software desarrollada en el entorno LabView
     para la calibración y control de rotadores ópticos motorizados.
   \item Una plataforma de software para la ejecución de rutinas de
     caracterización que involucran la medida de modulación de
     amplitud y de fase de un SLM ante múltiples estados de polarización.
  \end{itemize}
\item Un nuevo método para la caracterización de SLMs basado en
  funciones de minimización y una selección flexible de argumentos de
  entrada que aún no ha sido publicado.
\item Un programa, desarrollado en Python para la
  simulación de propagación de la propagación de estados de
  polarización representados como vectores de Jones a traves de
  matrices arbitrarias.  
\end{itemize}

% ----------------------------------------------------------------------

\section{Trabajo futuro}

Los métodos y herramientas desarrollados tanto en la caracterización
de elementos birrefringentes, como en la generación y caracterización
de VOs, permiten que los proyectos del Grupo de Óptica Aplicada cuenten
con las capacidades para adelantar la investigación en la construcción
de un microscopio de fase usando VOs y la corrección de aberraciones
para sistemas de escritura punto a punto en aplicaciones de
holografía. 

Como trabajo futuro se propone estudiar la generación y
caracterización, vía PD, de VOs en sistemas con iluminación parcialmente coherente

Adicionalmente, tenemos en curso investigaciones derivadas de la
caracterización de SLMs con las cuales se plantea hacer un sistema de
polarimetría-interferométrica para otros tipos de aplicaciones.



