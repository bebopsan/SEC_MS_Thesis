
% this file is called up by thesis.tex
% content in this file will be fed into the main document

%: ----------------------- introduction file header -----------------------

\chapter{Conclusiones y perspectivas}
\label{cha:Conclusiones}

% the code below specifies where the figures are stored
%\ifpdf
%    \graphicspath{{5_conclusion/figures/PNG/}{5_conclusion/figures/PDF/}{5_conclusion/figures/}}
%\else
 %   \graphicspath{{5_conclusion/figures/EPS/}{5_conclusion/figures/}}
%\fi


En el marco de esta tésis se implementó un sistema para la generación
de vórtices ópticos por medio de SLMs de trasmisión y un método para
la caracterización y corrección de aberraciones ópticas en haces Laguerre-Gausianos. 
La combinación de ambos cumple con los objetivos del proyecto de grado
y de los proyectos internos del grupo de investigación en óptica
aplicada. Y le deja al laboratório de Fotónica una herramienta para el
desarrollo de técnicas de micoscopía basadas en iluminación con haces
portadores de OAM. 
Específicamente. 

\begin{itemize}
\item Se mostró el resultado de una labor investigativa con la cual
  fue posible establecer un marco conceptual y teórico para la
  caracterización y puesta a punto de un modulador espacial de luz de
  trasmisión basado en cristales líquidos del tipo twisted nematic. 
\item Se presentó un sistema automatizado para la caracterización de
  pantallas de cristál líquido que se compone de una parte física, que
  involucra cuatro rotadores ópticos mecatrónicos, y una parte de
  software que adquiere los datos y los procesa para obtener las
  matrices de Jones que describen el elemento birrefringente para cada
  nivel de gris. 
\item Asimismo, se desarrolló una aplicacion de software en la
  plataforma Matlab $\circledR$ para la
  generación de máscaras de fase arbitrarias a ser proyectadas en el
  SLM. Esta aplicación permite:
  \begin{itemize}
    \item Crear máscaras de fase espiral de carga entera arbitraria
      sumadas a:
      \begin{itemize}
        \item Lentes.
        \item Rejillas de difracción de varios tipos.
        \item Aberraciones ópticas compuestas a partir de polinomios
          de Zernike.
      \end{itemize}
    \item Discretizar las máscaras de fase en la cantidad de niveles
      deseados, y asignando valores predeterminados a cada uno. 
  \end{itemize}
\item Por otra parte, se propuso un método novedoso para la
  caracterización de SLMs basado en el análisis de desplazamiento de
  franjas en un interferómetro con brazos que no comparten el mismo
  estado de polarización. Este método ha sido demostrado en
  simulaciones y nos encontramos en el proceso de corroboración
  experimental para validarlo. 
 \item Utilizando una configuración de estados de polarización que
   producen alta modulación de fase y baja modulación de amplitud se
   generaron VO en un sistema óptico 4F y se detectó que aún con buena
   modulación no se corrigen del todo las aberraciones. 
\item Dado lo anterior se implementó un método traído de aplicaciones
  en astronomía para la detección y
  corrección de aberraciones ópticas en haces con vorticidad óptica.
\item Este método fue validado mediante numerosas simulaciones, y
  experimentos. Concluimos que hemos desarrollado la capacidad para
  detectar y corregir aberraciones ópticas en sistemas ópticos
  iluminados con haces portadores de OAM.
\end{itemize}


% ----------------------------------------------------------------------


