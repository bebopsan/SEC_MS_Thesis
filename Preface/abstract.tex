
% Thesis Abstract -----------------------------------------------------


%\begin{abstractslong}    %uncommenting this line, gives a different abstract heading


\begin{resumen}        %this creates the heading for the abstract page
% Pon tu resumen aquí.

La generación de haces de luz con vorticidad óptica usando moduladores
espaciales de luz, es una capacidad que desde hace 
aproximadamente 10 años le ha
permitido a laboratorios de alto prestigio generar
innovaciones tecnológicas que han revolucionado campos de la
biomedicina. Entre ellas se encuentran las pinzas
ópticas para micromanipulación de objetos microscópicos, y nuevas
técnicas de microscopía de fase mejoradas con vórtices ópticos (VOs).  
En este trabajo se presentan los resultados de un proyecto del Laboratorio de Fotónica del Grupo de Óptica Aplicada de
la Universidad EAFIT, en el cual logramos implementar las herramientas y métodos
necesarios para la generación de VOs de calidad suficiente para la
investigación en estas áreas. 
En primera instancia, se desarrolló y caracterizó un montaje experimental
en el cual los VOs son generados al proyectar máscaras de
fase sobre un modulador espacial de luz de transmisión. Esto implicó
comprender el funcionamiento de este tipo de dispositivos e
implementar técnicas y herramientas para su caracterización, que garantizaran una
modulación de fase y amplitud adecuada. Una vez cumplido el objetivo de generar VOs
de forma experimental, se desarrolló un método novedoso de reconstrucción de fase basado en
la técnica de Diversidad de Fase (PD) para la caracterización de las
aberraciones ópticas del sistema que distorsionan sus distribuciones de intensidad. La
novedad del método resulta del uso de máscaras de fase espiral como
diversidades de fase, de tal forma que los mismos VOs distorsionados sirven
para la detección de las aberraciones que los están deformando. 
Finalmente, se presentan VOs de calidad, generados experimentalmente, y en los cuales
se ha introducido y corregido una aberración conocida como
demostración de las capacidades adquiridas.
\end{resumen}



%\end{abstractlongs}


% ---------------------------------------------------------------------- 
