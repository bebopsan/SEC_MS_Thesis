% this file is called up by thesis.tex
% content in this file will be fed into the main document

% Glossary entries are defined with the command \nomenclature{1}{2}
% 1 = Entry name, e.g. abbreviation; 2 = Explanation
% You can place all explanations in this separate file or declare them in the middle of the text. Either way they will be collected in the glossary.

% required to print nomenclature name to page header
\markboth{\MakeUppercase{\nomname}}{\MakeUppercase{\nomname}}

% ----------------------- contents from here ------------------------
%

%
%
%% acronyms


\nomenclature{AIV}{Armado, Integración y Verificación}
\nomenclature{ESPI}{Interferometría de Patrones de Speckle Elecrónica}
\nomenclature{SLM}{Modulador Espacial de Luz}
\nomenclature{SH}{Shack-Hartmann}
\nomenclature{LC}{Cristal Líquido}
\nomenclature{CCD}{Dispositivo de Carga acoplada}
\nomenclature{RMSE}{Error cuadratico medio}
\nomenclature{SNR}{Coeficiente señal-ruido}
\nomenclature{CTE}{Coeficiente de Expansión Térmica}
\nomenclature{HDR}{Alto Rango Dinámico}
\nomenclature{TVC}{Cámara de Termo-Vacío}
\nomenclature{FEA}{Análisis de Elementos Finitos}
\nomenclature{FEM}{Modelo de Elementos Finitos}
\nomenclature{TUM}{Technische Universit\"{a}t M\"{u}nchen - Universidad Tecnológica de Munich }
\nomenclature{OAM}{Momento Orbital Angular}
\nomenclature{VCSEL}{Láser Emisor de Superficie de Cavidad Vertical}
\nomenclature{WCOG}{Método de Compensación del Centro de Gravedad}
\nomenclature{IWCOG}{Método Iterativo de Compensación del Centro de Gravedad}
\nomenclature{FWHM}{Anchura Total a la mitad del Máximo}
\nomenclature{SPT}{Transformada Espiral de Fase}
\nomenclature{LK}{Método Lucas-Kanade}
\nomenclature{LINES}{Laboratorio de Instrumentación Espacial}
\nomenclature{INTA}{Instituto Nacional de Técnica Aeroespacial}
\nomenclature{LCVR}{Retardadores Variables de Cristal Líquido}
\nomenclature{FDT}{Full Disk Telescope - Telescopio de Disco Completo}
\nomenclature{HRT}{High Resolution Telescope - Telescopio de Alta Resolución}
\nomenclature{RFM}{Refocus Mechanism}
\nomenclature{PHI}{Polarimetric and Helioseismic Imager}
\nomenclature{BS}{Divisor de haz - Beam Splitter}
\nomenclature{IAC}{Instituto de Astrofísica de Canarias}
\nomenclature{GTC}{Gran Telescopio de Canarias}
\nomenclature{ESA}{Agencia Espacial Europea}
\nomenclature{PMP}{Polarization Modulation Package - Paquete de Modulación de la Polarización}
\nomenclature{CMOS}{Semiconductor Complementario de Óxido Metálico}
\nomenclature{APAN}{Nemático Antiparalelo - Anti-Parallel Nematic}
\nomenclature{LBD}{Globos de Larga Duración - Long Duration Balloon}
\nomenclature{NASA}{Administración Nacional de Aeronáutica y Espacio de USA - National Aeronautics and Space Administration}
\nomenclature{GACE}{Grupo de Astronomía y Ciencias del Espacio}
\nomenclature{IAA}{Instituto de Astrofísica de Andalucía}
\nomenclature{FOV}{Campo de Visión - Field of View}
\nomenclature{PSF}{Función de Punto Esparcido - Point Spread Function}
\nomenclature{WFE}{Error del frente de onda - Wavefront error}
\nomenclature{MTF}{Modulación de la función de transferencia óptica - Modulation Transfer Function}
\nomenclature{PD}{Diversidad de Fase - Phase Diversity}
\nomenclature{WHT}{William Herschel Telescope}
\nomenclature{OGS}{Optical Ground Station}
\nomenclature{FPGA}{Field-programmable gate array}
\nomenclature{ENO}{Observatorios Europeos del Hemisferio Norte - European Northern Observatory}
\nomenclature{AO}{Óptica Adaptativa - Adaptive Optics}
\nomenclature{GS}{Método de Gerchgerg-Saxton}
\nomenclature{SBMIR}{Single-beam Multiple-Intensity Reconstruction Technique}
\nomenclature{OTF}{Función de transferencia óptica - Optical transfer function}
\nomenclature{ASD}{Descomposición espectral angular - Angular Spectrum Descomposition}
